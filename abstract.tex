
As one of the most economically significant and fastest growing sectors of the Internet, home networks availability, performance and reliability have attracted interest from researchers. Conducting network measurements can give us insight into home networks, but doing so requires careful consideration of user privacy and efficient use of limited computational resources. 

To better understand the characteristics of home networks, we developed SOAR, an open research testbed that utilizes computational resources provided by end users on their home wireless routers with custom firmware. Unlike most other platforms, \sysname provides a privacy protection of embedded device data and maintains the security of donated device from potentially buggy experiment codes. We find that our platform is flexible enough to implement a variety of network measurements (e.g., the set of devices, aggregate traffic statistics, characteristics of the WiFi environment, etc) despite its security restrictions. This paper discusses some of the challenges we faced building and using a platform for deploying measurement in home networks, describe its design and implementation and illustrate how it protects user privacy and manages experiments efficiently. We demonstrate SOAR's utility with three case studies: comparing the wireless performance of 2.4 GHz and 5 GHz bands, studying channel survey statistics and home wireless network environment. We believe that \sysname provides a valuable insight of home wireless networks and will help researchers to understand home networks better.
\end{abstract}
