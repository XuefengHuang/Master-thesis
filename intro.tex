\chapter{INTRODUCTION}
\label{sec.introduction}
Broadband Internet access for home use is rapidly evolving. The United States
alone has more than 279 million broadband users. The number of Internet 
users in other regions is even more impressive, with China counting more 
than 641 million~\cite{asia}. Yet, despite its pervasiveness, little is 
known about most home network's availability, performance and reliability. This lack of knowledge hampers progress in a number of important research 
areas, from ISP performance to large-scale
topology mapping and wireless network utilization. Researchers, policymakers 
and Internet Service Providers (ISPs) are also eager to learn more about 
emerging technologies made possible by these networks. ``Smart homes," in 
which devices from thermostats to door locks can be controlled remotely are
 now exciting realities. In 2016, 4 billion new objects (e.g., laptops, 
handhelds, interactive television, wireless based security cameras) will 
become available to consumers~\cite{gartner}. But with the benefits of these 
products also come concerns. For these devices to work, they usually need to 
be connected to a wireless router. If a router is compromised via buggy 
software, it is possible that hackers could unlock not only critical data, 
but also someone's front door. Therefore, as their use grows, understanding 
vulnerabilities in both routers and broadband networks becomes crucial in 
order to prevent these types of attacks.

Due to these challenges, the value of studying home networks for researchers, 
policy makers and ISPs continues to grow. At this point in time studying 
home networks on a large scale has been difficult because network 
technologies, such as network address translators (NATs), present only a 
partial view of the home network to the global Internet. To better understand 
the behavior of networks, a formal study using an experimental platform that 
can gather data from homes is needed to provide visibility into the 
unseen sections of the network. Such a platform should also provide a set of 
programming interfaces to support as wide an array of measurements as 
possible, including factors affecting broadband performance, or data that 
can increase understanding of usage and connectivity in home networks.
Lastly, the platform proposed here must ensure the privacy of the 
user and prevent abuse of his or her device or network resources.

A number of measurement and experimentation platforms have been developed to 
support controlled network experimentation and broadband characterization 
for home networks, including gateway-based platforms~\cite{bajpai2014lessons,
samknows,183951,yiakoumis2014behop} and browser or host-based platforms~\cite{
dhawan2012fathom,kreibich2010netalyzr,sanchez2014measurement}. Host-based 
platforms run as applications on the client side and typically suffer from a 
few limitations (e.g., reflecting the network performance of the host or 
application rather than of the network itself, providing only one-shot 
measurements rather than continuous measurements). A gateway-based platform, 
on the other hand, enables continuous measurements and collects network 
feedbacks from all the WiFi devices in the home networks. However, these 
platforms can not balance the trade-off between functionality and security.
 For instance, the process of vetting BISmark~\cite{183951} experiments for 
researchers is manual, which will be a limiting factor as deployment grows. 
Inspired by these projects, we decided to design a gateway-based platform 
that enables continuous, comprehensive and direct measurements. We also 
designed the platform with user security and privacy as first-order concerns.
 
In this paper, we introduce \sysname, a distributed cloud platform that 
allows researchers to run both active and passive experiments securely from 
a wireless router on a home network. Through a programmable interface on the 
device, the platform enables researchers to deploy a wide range of network 
measurements from a vantage point between the access ISP and the home 
network, as shown in Figure~\ref{figure:design}. Researchers can access and 
compile data on systems worldwide, while preserving the user's privacy, and 
securing the device itself. The information gathered from such studies could 
assist users in evaluating the services they are paying for, researchers in 
understanding the factors impacting performance, and ISPs in finding ways to 
improve their service. Because it is designed for running measurement 
software on a home access point, a platform like \sysname has a number of 
unique strengths and challenges. First, software must be small and easy-to-
deploy because \sysname is deployed on resource-constrained devices. Second, 
\sysname sits on the direct path of real Internet users. Any buggy or 
malicious experiments could be able to disrupt Internet connectivity. Thus 
it must guarantee that a poorly designed experiment can not negatively 
affect a user's Internet connection. Third, it is important to make the 
system robust by providing remote maintenance and updates because of the 
unmanaged complexity of its home network environments. Finally, \sysname 
must provide a rich set of APIs for a variety of network measurements. 

\begin{figure}%[h]
\centering
\includegraphics[width=0.8\columnwidth]{figure/home-network.png}
\caption{The \sysname-enabled router sits on the direct path of real 
Internet users. It supports both active and passive network measurements.}
\label{figure:design}
\end{figure}


\sysname is based on Seattle testbed, a community-driven and open-source cloud computing system~\cite{zhuang2013experience,cappos2009seattle}. Compared to computer and mobile device environments, deployment on home wireless routers has more resource limitations, such as restricted computational resources. However, recently we have been able to port Seattle Testbed to OpenWrt, a popular Linux platform for home routers~\cite{openwrt}. Users can build their own \sysname installer (IPK) via a config file we provide using OpenWrt SDK and install it on the device directly. 


\sysname supports standard implementations of useful measurement primitives such as ping, traceroute, network traffic capture, and the like. We extend eight functionalities based on Repy (the core sandbox of Seattle), include 
\texttt{get\_network\_
bytes}, \texttt{get\_network\_packets}, \texttt{get\_network\_interfaces}, \texttt{wifi\_status}, \texttt{scan}, \texttt{get\_
station}, \texttt{ping} and \texttt{traceroute}. This rich set of measurement primitives help researchers to implement a wide range of network measurements, such as mapping Internet paths (via traceroute), studying home network usage patterns and understanding wireless network performance.

\sysname supports a constrained environment of programmable measurements. In order to securely interact with home routers on remote user devices, we use Fence (a non-intrusive mechanism that mediates and limits access to diverse resources using uniform resource control)~\cite{li2015fence} to allocate a fixed percentage of the device's CPU, memory disk, and other resources to one or more VMs. For example, we set the legal times of accessing the \texttt{/proc} file system to prevent DoS attacks using our API calls. To prevent exfiltration of sensitive information from devices, we use security layers to mediate access to the wireless sensor by restricting capability access and reducing the precision of returned data. 


By integrating \sysname into home networks, we get the benefits of a real world deployment while ensuring flexibility to run experiments without compromising home networks. We demonstrate \sysname's utility by implementing both active and passive measurements that together exercise different new API calls. We monitor our lab's network traffic, which is in an office building, from the vantage point of the access point. We report on additional experiences gained using \sysname in three different use cases. We find that there are many factors affecting throughput on 2.4 GHz and 5 GHz band. We also find that most of the access points select non-overlapping channels (e.g., channel 1, 6, 11) to avoid adjacent-channel interference. 


The primary contributions of this study are as follows:
\begin{itemize}
\item We introduce \sysname as a platform for experimentation on home wireless routers that provide a programmable sandboxed environment and dynamic access control mechanism for router security and privacy. Our design ensures experiment code cannot harm the devices of users and reduces the risk of privacy problems through the use of security layers.
\item We implement new research capabilities for home wireless routers by improving on the Seattle sandbox. This rich set of measurement primitives help researchers to implement a wide range of network measurements, such as mapping Internet paths (via traceroute), studying home network usage patterns and understanding wireless network performance.
\item To our knowledge, we are the first wireless router testbed to provide both security and performance isolation using a sandbox environment. 
\end{itemize}


The rest of this thesis is structured as follows: We provide related work and motivation in \S{\ref{sec.relatedwork_motivation}}. In \S{\ref{sec.goals_challenges}}, we point out the goals and challenges of our work. \S{\ref{sec.design}} and \S{\ref{sec.implementation}} describes the design and implementation of \sysname and characterizes our current deployment. Then we present three study cases to illustrate the benefits of an experimental platform that can run on a home wireless router in \S{\ref{sec.evaluation}}. Finally, we discuss our conclusions in \S{\ref{sec.conclusion}}. 