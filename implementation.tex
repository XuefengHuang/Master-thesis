\section{Implementation} 
\label{sec.implementation}

Based on the design introduced in \S{\ref{sec.design}}, we implement a 
secure network testbed. We describe the implementation details of our system 
in this section.

\subsection{Architecture of Seattle}
The primary goal of Seattle is to run a wide range of network measurements 
on remote home gateway without compromising home network security and privacy
. To achieve this goal, we try to minimize the disruption of Internet 
connectivity. From distributed platform design, we leverage Seattle Testbed 
which is a community-driven, open source cloud computing system to acheive 
this goal. 

When distributed platform is achieved, we also want to run arbitrary codes 
across donated devices. We safely extend new research capabilities in our 
Repy sandbox. We use Repy sandbox to reconstruct a programming interface, 
which provides network functionalities sufficient for most network 
measurements.

\subsection{Implementation Details}
\textbf{Seattle porting onto home wireless router:} 
Seattle is a platform for networking and distributed systems 
research. It supports a wide range of devices including desktop, laptop, 
servers, etc.

Compared to desktop and laptop environments, deployment on home wireless 
router has more resource limitation such as restricted computational 
resources. However, recently we are able to port Seattle to embedded devices.
Users can now build their own Seattle installer(IPK) via OpenWrt SDK and 
install it on the device easily. 

\textbf{Supporting wireless interface on wireless embedded devices:} 
Due to its isolation of the virtual machine, Seattle cannot normally access 
network traffic data on a user's home router devices, such as bandwidth, 
WiFi SSID and signal strength, etc. However, if the data is anonymized, it 
would be very useful to researchers. Seattle provides an API for reading 
wireless network data from a Seattle VM and providing it to users. It is 
extended from Seattle RepyV2 and provide system wide protection against 
greedy processes that include isolating performance. 

\textbf{Reference monitor:}
Seattle also provides a framework of reference monitor, to enforce mandatory 
access control to sensitive data in real time. Based on the type of network 
measurement experiment, the data returned by the embedded devices can be 
passed through, filtered, or dropped by the reference monitor.

\subsection{Deployment}
Deploying Seattle Testbed in a real-world setting is essential to study and 
evaluate bandwidth performance and wireless performance under the conditions 
found in a real network (client diversity, mobility, interference with 
neighboring networks). For our study, we deploy TP-Link TL-WDR3600 router, 
which has a 560 MHz MIPS processor, 8MB of flash storage, 128MB of RAM and a 
dual-band wireless interface. We replace the router’s default software with 
a custom version of OpenWrt Linux~\cite{openwrt}. Some users can download 
the Seattle package from our project page and install it on their own 
hardware via opkg package manager.

