\section{Threat Model}
\label{sec.threat_model}
While a wide ranges of programming functionalities on embedded devices are very useful, they also pose a risk to users. To understand the scope of issues that our work will address, we summarize all the security issues possible within the home network.

The first issue is insecure interfaces and application programming interfaces. Low-level operations (like writing to files or sending network messages) will be accessed through Application Programming Interfaces (APIs). Malfunctions and errors in the interface can lead to unwanted exposure of users data and attacks upon the data?s integrity. For example, attackers can use code injection attack to gather data on local host or gain complete control. Threats can also exist as result of  poorly designed or implemented security policy. If these policies can be bypassed, the platform can be easily abused by attackers. Regardless of the threat origin, APIs need to be made secure against accidental and malicious attempts to circumvent the APIs.

The second issue is denial of service. For a distributed system, it is necessary to control resource consumption on the local host, minimizing the impact of Internet connectivity. As such, if an attacker consumes a large amount of resources, it can lead to poor performance. Therefore, managing resources at large scale while providing performance isolation is a key challenge for any distributed system software.

The third issue is home network compromise. There is an important difference between Seattle Testbed and other testbeds, such as PlanetLab. Our experiments involve home network. Attackers can get the control rights of the home network via our platform and send the malicious control commands to all the home devices, which may lead to undesired consequences. By exploiting vulnerabilities in home networks, attackers also can gather information on targets, threatening home network?s privacy and safety, understanding behavior and patterns. For instance, some devices can be shut down unusually or the information of some devices can be monitored.

The fourth issue is data leakage. Here we consider the following problem: how does a user know whether their data is accessible and has not been corrupted? Currently, many testbeds do not explicitly guarantee data integrity or availability. Users have access to the hardware and can modify firmware via testbed. Attacker also could contribute malicious data to influence conclusions. These impose new security challenges.Thus, it must prevent an experiment from performing malicious actions like reading the user?s sensitive files or tampering system files.