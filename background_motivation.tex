\chapter{Background and Motivation}
\label{sec.background_motivation}
In this section, we present some background information on factors that shaped the design and deployment of \sysname. First, we provide the related works behind its design. Then, we come up with design goals for a home wireless router based network measurement platform. Next, we discuss some of the key issues and  challenges in designing and implementing such platform. Finally, we look at the threat model of a programming interface for home wireless router based network measurement platform.

\section{Related Work}
\label{ssec.related_work}
Our work shares goals with and builds upon ideas from several prior large scale platforms targeting home network research. In the following paragraphs, we briefly review related works on popular network testbeds and home network experiments. 

\textbf{Measurement platforms.}The BISmark~\cite{183951} project aims at understanding the performance of wired access networks in homes through the long-term deployment of gateways. We use a similar approach but focus on building a programming framework to balance between flexibility (allowing researchers to do a wide set of experiments) and a constrained programming environment (limiting researchers from doing experiments that could interfere with home users). The RIPE Atlas~\cite{ripeatlas} is a global network of probes that measure Internet connectivity and reachability, providing an unprecedented understanding of the state of the Internet in real time. It has deployed thousands of probing devices worldwide, but their capabilities are limited to simple measurements (e.g., ping, traceroute). Our testbed provide a wide range of capabilities that support the implementation of as wide an array of network measurements as possible. SamKnows~\cite{samknows} has designed and developed its performance tests in house, adhering to IETF RFCs where appropriate. All measurements are written in C, for performance and portability across a range of hardware platforms. But it cannot adapt their measurements procedure and software in the course of a measurement study. \sysname instead allows for very general programmability and rapid code updates on wireless router. BeHop~\cite{yiakoumis2014behop} is a wireless testbed for dense WiFi networks often seen in residential and enterprise settings. \sysname not only evaluate WiFi management strategy, but also study different aspects of home networks, including access link performance, user behavior patterns and topology and connectivity characterization. WiSe~\cite{patro2013observing} is a measurement and management infrastructure that uses APs as vantage points to collect comprehensive information about network. However, WiSe focuses solely on the wireless network properties and not on the rest of the ISP path. \sysname instead focus on both characterization of the wired Internet path from the ISP’s network into the home and the wireless network properties.

Prior works such as~\cite{sanchez2014measurement},~\cite{dhawan2012fathom},~\cite{kreibich2010netalyzr} has focused on end-host techniques for measuring end to end performance in home networks. Dasu~\cite{sanchez2014measurement} is a host-based software that analyzes BitTorrent traffic to characterize ISP performance. Because Dasu piggybacks on BitTorrent, it has a large user base. Fathom~\cite{dhawan2012fathom} is a Firefox extension that provides a programmable interface for writing and launching measurements from the convenience of the web browser. Netalzyr~\cite{kreibich2010netalyzr} also lets users conduct a series of tests from a browser. However, these testbeds cannot support for continuous measurements (i.e., since hosts can be turned off, moved, etc.). In addition, they do not reflect the performance of a fixed network vantage point due to the limitation of host. Our work extends on Seattle Testbed\cite{cappos2009seattle}, which runs on laptops, tablets and smartphones. In contrast to Seattle Testbed, our platform is deployed in home networks on resource-constrained devices and extends eight capabilities based on Seattle sandbox.

Most active measurement and experimentation research relies on dedicated infrastructures (PlanetLab~\cite{chun2003planetlab}, MLab~\cite{mlab}). However, such infrastructures support long-running measurements at cost of limited vantage point diversity because nodes primarily deployed in well-provisioned academic or research networks which are not representative of the real users' Internet. In \sysname, experiments can observe all traffic passing through home wireless router. Scriptroute~\cite{spring2003scriptroute} added the capability for any researchers to run experiments using scripts in a sandboxed environment in PlanetLab. We share Scriptroute's goal of building a community platform for network measurements. However, \sysname is intended for large scale deployment on home wireless routers brings new challenges in particular: resource limitations and home network security.

\textbf{Measuring home network performance.} There has been significant interest in measuring home networks. Previous studies \cite{aguayo2004link}\cite{kotz2005analysis}\cite{raman2009feasibility} have evaluated the performance of 802.11 wireless networks in campus, enterprise, urban and rural environments. For example, \cite{raman2009feasibility} undertakes a detailed measurement study for mesh networks in rural locations, it provides evidence to support a conclusion that external interference is the main cause of unpredictable link behaviour. \cite{kotz2005analysis} conducts a trace-based study of WLAN users in Dartmouth College campus, in an effort to understand patterns of activity in a network.

Recent studies have tried to measure home networks in various ways, from the end-host \cite{chetty2011my}\cite{dicioccio2012probe}\cite{sanchez2013trying}, or from gateway routers \cite{grover2013peeking}\cite{sundaresan2015measuring}\cite{patro2013observing}\cite{sundaresan2013measuring}\cite{papagiannaki2006experimental}\cite{pefkianakis2015characterizing}. For example in \cite{sundaresan2015measuring}, authors deploy BISmark on commodity routers in 66 homes to identify the relationship between wireless metrics and TCP performance of user traffic. \cite{patro2013observing} deploys WiSe to perform measurements of wireless properties in a wide diversity of homes and presents a simple metric to estimate TCP throughput in their deployments. \cite{sanchez2013trying} studies the complexity of home networks around the world through UPnP measurements collected from Dasu. \cite{sundaresan2013measuring} deploys FCC/SamKnows and BISmark to characterize performance to nine popular Web sites from 5,556 access networks and identify factors that create Web performance bottlenecks. We share \cite{sundaresan2015measuring}'s idea of studying how home wireless performance characteristics affect the performance the user traffic in real homes. However, we also study wireless performance in concert with airtime utilization.



\section{Goals}
\label{ssec.goals}
We identify the design goals for network measurement platform in home networks:
\begin{itemize}
\item Safety: Internet users that participate in the testbed should not face significant risk. Code should be strictly sandboxed. Code executed on the testbed must not interfere with the performance or correctness of user's Internet connection. As mentioned in \S{\ref{ssec.threat_models}}, there are a few potential risks in home networks, therefore we design a network measurement platform with user security and privacy as a first-order concern.

\item Privacy: Our goal is striving to maximize utility to experimenters while not compromising the privacy of the user (e.g., expose home network usage behaviour).

\item Flexibility: The related works in \S{\ref{ssec.related_work}} show that a good network measurement platform should support for both passive and active measurements. Passive measurements provide precise and continuous understanding of network and active measurements provide some insight into the way real network traffic is treated within the network. We therefore require a wide range of APIs that support the implementation of as wide an array of network measurements as possible.

\item Lightweight: We desire the platform's weight to be as light as possible for two reasons: (i) Due to \sysname is deployed in home networks on resource-constrained devices, it is impossible to run heavy softwares. (ii) \sysname nodes are on the direct path of real Internet users, hence we don't want to impact users' Internet connectivity.

\item Accuracy: We desire the measurement platform to accurately track 
general network information. Precise experiment data is able to help researchers study and understand network better.

\item Easy to install and use: Unlike computer or mobile environments, home wireless routers are not familiar to most people. Therefore, installing is a big challenge. We require to provide an installer to help users to install, un-install, and stop easily.
\end{itemize}

\sysname's design (\S\ref{sec.design}) addresses the first three goals and its implementation (\S\ref{sec.implementation}) the remaining goals.

\section{Challenges}
\label{ssec.challenges}
A testbed that allows researchers to execute code across donated embedded devices faces four major challenges: 
\begin{itemize}
\item First, the embedded devices poses a limited resources challenge. It is hard to run heavy scripting languages like Python or Ruby. Unlike host-based or fixed server network testbed, router-based network testbed deploy on resource-constrained devices which have too little RAM and flash storage. Therefore, our platform must be small and easy-to-deploy. To overcome this challenge, BISmark uses standard UNIX utilities and small C programs with shell and Lua scripts.  
\item The second challenge is user's network experience. \sysname is on the direct path of real Internet users. A malicious network measurement experiment will noticeably affect a normal user's network experience by consuming too much network resources. Therefore, our platform must ensure that a poorly designed experiment cannot disrupt a normal user's Internet connectivity. It is also a challenge for BISmark. To fix this issue, BISmark has to review each experiment codes manually. Dasu uses a sandboxed environment for safe execution of experiment and limits resource consumption. However, it uses Drools so that we need to port from Java.
\item The third challenge is to reconcile the need for open data with user privacy. Since home router connects to many devices in home networks, they can generate valuable data for home network research study. However, these valuable data can also threaten user privacy. For example, network traffic statistics can yield insight into user behaviour and then indicate when users are home and using the network. Due to these concerns, network testbed either do not collect data from real Internet users or use data collected from a controlled group of participants. Home users want information security and researchers want to gather data and information. Therefore, it is difficult to balance data collection with privacy protection.
\item Further challenges remain regarding software upgrades. Unlike PlanetLab nodes deploy in universities, network testbed in home networks cannot get enough technical supports. In addition, OpenWrt's built-in \texttt{opkg} lacks features for managing software in the homes of non-technical users. Therefore, software management is a big challenge for \sysname.  
\end{itemize}
As the development of hardware technology and based on our previous practical experience, running Python codes are going well on embedded devices. We have ported \sysname to home wireless router in \S{\ref{ssec.deployment}} and ran many network measurements (including multithreading, network programming, file handling and so on) in \S{\ref{sec.evaluation}}. And the second and third challenges can be solved via using a secure and performance-isolated sandbox (\S{\ref{sec.sandbox}}) and a reference monitor framework  (\S{\ref{sec.policy}}) we proposed. Finally, \sysname provides a software updater (\S{\ref{sec.softwareupdater}}) to make sure system robustness.

\section{Threat model}
\label{ssec.threat_models}
While a wide ranges of programming functionalities on embedded devices are very useful, they also pose a risk to users. To understand the scope of issues that our work will address, we summarize all the security issues possible within the home networks.

\begin{itemize}
\item The first category is insecure interfaces and application programming interfaces. Low-level operations (like writing to files or sending network messages) will be accessed through Application Programming Interfaces (APIs). Malfunctions and errors in the interface can lead to unwanted exposure of experimenters' data and attacks upon the data's integrity. For example, attackers can use code injection attack to gather data on local host or gain complete control. Threats can also exist as result of poorly designed or implemented security policy. If these policies can be bypassed, the platform can be easily abused by attackers. Regardless of the threat origin, APIs need to be made secure against accidental and malicious attempts to circumvent the APIs. In previous works, to avoid potential security risks (e.g., altering your host filesystem), Dasu\cite{sanchez2014measurement} enables programmable experiments without requiring root access. 

\item The second category is denial-of-service attacks. For a distributed system, it is necessary to control resource consumption on the local host and minimize the impact of Internet connectivity. As such, if an attacker consumes a large amount of resources, it can lead to poor performance. Therefore, managing resources at large scale while providing performance isolation is a key challenge for any distributed system software. Many network testbeds consider this issue when designing their platform. For instance, all experiments on Scriptroute\cite{spring2003scriptroute} are executed in a resource-limited sandbox to protect platform from resource abuse and launching denial-of-service attacks. Dasu also uses a sandbox environment for defining and deploying experiments, and controlling their malicious impact.

\item The third category is home network compromise. There is an important difference between \sysname and other testbeds, such as PlanetLab, MLab. Our experiments involve home networks. Attackers can get the control rights of the home network through our platform and send the malicious control commands to different kinds of WiFi enabled devices such as laptop, printers, wireless based security cameras, entertainment devices and other security systems, which may lead to undesired consequences. By exploiting vulnerabilities in home networks, attackers also can gather information on targets, threatening home network's privacy and safety, understanding behaviour and patterns. For instance, some devices can be shut down unusually or the information of some devices can be monitored.

\item The fourth category is data leakage. Here we consider the following problem: how does a user know whether their data is accessible and has not been corrupted? Currently, some network testbeds (e.g., ~\cite{183951}) do not explicitly guarantee data integrity or availability. Experimenters have access to the hardware and can modify firmware through testbed. Therefore, attacker has ability to contribute malicious data to influence conclusions. These impose new security challenges. Thus, it must prevent an experiment from performing malicious actions like reading the user's sensitive files or tampering system files.
\end{itemize}

In the following section, we introduce our proposed scheme (\S{\ref{sec.security}} and \S{\ref{sec.privacy}}) that is able to detect and defend against all these four categories of threats above.