\section{A New Design for Building Secure Network Testbed}
\label{sec.design}
Our goal is to build a secure network testbed that can mitigate the problem 
of end user's security and privacy issue. 

The previous works introduced in \S{\ref{sec.related_work}} serves the 
purpose of understanding the disadvantages of current network testbed. Our 
finding in \S{\ref{sec.related_work}} suggest that it is important to 
balance the trade-off between security and functionalitiy.

In this section, we use this idea to guild our new design for building 
secure network testbed. Our goal is to allow researchers to run texperiments 
while reducing the risk of security and privacy problem.

\subsection{Overview}
Today's poor visibility into the home network hampers progress in a number 
of important research areas, from WiFi management strategies to broadband 
characterization. Previous works~\cite{sanchez2014measurement}~\cite{
dhawan2012fathom} usually provides a programmable interface for writing and 
launching measurements from the perspective of end systems. Our proposed 
approach in supporting a wide range of measurements is using a programmable 
interface as well. Unlike such prior network measurement efforts, our work 
places network measurement capabilities embedded into home gateway. This 
approach has various unique advantages: (i) our platform is able to observe 
all traffic to and from its clients without missing any network traffic. (ii)
home gateway is “demarcation point” between the home wireless network and 
the access link so that it is the best place to learn about wired and 
wireless network. Although BISmark~\cite{183951} has been worked on 
deploying measurements  through the long-term deployment of gateways in 
homes, it has several security issues that might cripple a user\’s Internet 
connection. Seattle instead allows untrusted parties to run arbitrary codes 
on our platform via a programmable interface and resource protection 
framework. A user can let a third party have access to their network traffic 
data within a security and performance isolated virtual machine. Users not 
only are able to control how much information they would share with the rest 
of world, but also decide how many resources an application can use. For 
example, an experiment could be prevented when consuming so much resources 
or access some interfaces which are not allowed by device\'s owner. 

Seattle is composed of three separate parts: router owner, clearinghouse and 
researcher. Router owner donates computational resources to our platform, 
clearinghouse enable researchers to pool and share resources, researcher is 
seeking to run experiments on remote router.

Before conducting any network measurement experiments, owner first downloads 
installer package from clearinghouse which is a testbed server that keep 
track of routers and allow researcher to access available routers. To run 
codes on owner's router, researchers needs to download an experiment manager 
to his own laptop. Researcher uses the experiment manager to access the 
remote router directly, upload experiment codes and start or stop the 
execution of the experiment.

\subsection{API Design}
Our main goal is to provide a wide range of capabilities that support the 
implementation of a broad range of network measurements on home gateway. In 
order for us to perform and collect measurements from home gateway, we have 
proposed and implemented a programmable API that expose a set of simple 
capabilities. Our experiment management tool is able to use this API to 
perform a rich set of network measurements. Seattle Testbed currently 
provides three main API families and supported the types of network 
measurement (listed in Table 1 and Table 2):

\begin{table*} %[!ht]
\scriptsize
\centering
\begin{adjustwidth}{1cm}{-1cm}
\begin{tabular}{|p{.3\textwidth}| p{.5\textwidth}| m{.7\textwidth}|}
\hline
\textbf{API}    &  \textbf{Description} \\
 \hline
 {\bf system.scan} & {\bf Collect the list of access points found with a WiFi scan. For each access point we collect BSSID, SSID, signal strength and channel number.} \\
\hline
 {\bf system.get\_station} & {\bf Record downlink statistics per associated client (e.g., Total packets sent, received, retried, client's signal strength at home wireless router).} \\
\hline
 {\bf utils.get\_network\_interface} & {\bf Return a list of available network interfaces.} \\
\hline
 {\bf utils.get\_network\_bytes} & {\bf Record information about the configured network interfaces. The statistics include metrics such total number of received or transmitted bytes, drops, errors.} \\
\hline
 {\bf utils.get\_network\_packets} & {\bf Record information about the configured network interfaces. The statistics include metrics such total number of received or transmitted packets, drops, errors.} \\
\hline
 {\bf network.ping} & {\bf A pure python ping implementation using raw sockets.} \\
\hline
 {\bf network.traceroute} & {\bf Return the route packets take to network host. } \\
\hline
 {\bf network.tcp} & {\bf Send or receive TCP traffic.} \\
\hline
 {\bf network.udp} & {\bf Send or receive UDP traffic.} \\
\hline
 {\bf network.getmyip} & {\bf Returns the localhost's \"Internet facing\" IP address.} \\
\hline
 {\bf network.dns} & {\bf Perform DNS lookups} \\
\hline
\end{tabular}
\end{adjustwidth}
\caption {API design}
\label{table:api_design}
\end{table*}

\begin{table*} %[!ht]
\scriptsize
\centering
\begin{tabular}{|l|c|c|c|c|c|c|}
\hline
\textbf{}    &  \textbf{ISP performance} & \textbf{Home network usage}  & \textbf{Internet connectivity and reachability}\\
 \hline
 {\bf system.scan} & {\bf *} & {\bf } & {\bf }\\
\hline
 {\bf system.get\_station} & {\bf *} & {\bf *} & {\bf }\\
\hline
 {\bf utils.get\_network\_interface} & {\bf *} & {\bf } & {\bf }\\
\hline
 {\bf utils.get\_network\_bytes} & {\bf *} & {\bf } & {\bf }\\
\hline
 {\bf utils.get\_network\_packets} & {\bf *} & {\bf } & {\bf *}\\
\hline
 {\bf network.ping} & {\bf *} & {\bf } & {\bf *}\\
\hline
 {\bf network.traceroute} & {\bf } & {\bf } & {\bf *}\\
\hline
 {\bf network.tcp} & {\bf *} & {\bf } & {\bf *}\\
\hline
 {\bf network.udp} & {\bf *} & {\bf *} & {\bf }\\
\hline
 {\bf network.getmyip} & {\bf *} & {\bf *} & {\bf *} \\
\hline
 {\bf network.dns} & {\bf *} & {\bf } & {\bf *}\\
\hline
\end{tabular}
\caption {Network measurement type}
\label{table:type}
\end{table*}

\emph{system} provides a python interface to low-level system calls on system
. Seattle Testbed sanitizes the call arguments, parsing complex output and 
return results to the caller. Examples in this family include get\_station 
and scan implementation. These methods use the iw command-line tool 
to gather all this information. Researchers are able to study dense WiFi 
networks by these functions.


\emph{utils} records aggregate network statistics for passively collected 
data. These methods read directly and parse data from \emph{/proc/dev/net}. 
These are able to leverage the naturally-generated network traffic as 
passive measurements (particularly in the area of broadband characterization)
by continuously monitoring the home gateway Internet connection.

\emph{network} provides low-level measurements functionalities that can be 
combined to build a wide range of measurement experiments. Current available 
measurement functions include ping, traceroute, performing DNS lookups, 
obtaining the local IP address, and sending or receiving TCP and UDP traffic.

\subsection{Security}
Seattle nodes are on the direct path of real Internet users, therefore we 
must ensure Seattle does not allow to risk to the host and network when 
executing experiments. To do so, Seattle Testbed uses a sandboxed 
environment for safe execution of external code, performance isolation for 
programs and traffic containment service. 

Safe execution: This is achieved by Repy, which is a subset of the POSIX API 
constructed with the Restricted Python sandbox. To prevent insecure actions 
the sandbox hooks into the Python parser and reads the program's parse tree. 
Only actions that the sandbox can verify as safe may execute. Repy is highly 
portable and has been running on a variety of different operating systems 
over the past five years.

Limit on resource consumption: Seattle carefully control resource 
consumption on the host and minimize the impacts on the user's Internet 
connection. In order to securely interact with home gateway on remote user 
devices, Seattle uses Fence~\cite{li2015fence} to allocate a fixed 
percentage of the device's CPU, memory disk, and other resources to one or 
more VMs. As a result, experiment programs are sandboxed and securely 
isolated from other programs on the same device. 

Traffic containment service: Due to seattle allows TCP/UDP socket operation, 
attackers can exploit the security vulnerability to compromise the 
computers, printers and other connected devices in home networks. To prevent 
such actions, there must be restrictions on which hosts testbed nodes can 
communicate with. Seattle provide a interposition mechanism, which restricts 
communication between the local node and a remote node based on the 
destination IP, port, or some combination of the two. 

\subsection{Privacy}
Although a rich set of capabilities enhance the convenience of user 
interfaces and application usefulness, they also raise serious privacy 
concerns. For example, network traffic volume over interface and the 
information of connected devices in home networks produce a rich history, it 
will lead to behaviour prediction, such information invades user privacy. 

In order not to allow measurements to exfiltrate sensitive information from 
home gateway, Seattle provides a framework of reference monitor~\cite{ref}, 
to enforce mandatory access control to sensitive data in real time. Based on 
the type of network measurements, the capabilities can be passed through, 
filtered, or dropped by the reference monitor. For a user who wants to learn 
about Internet censorship, a filter might perform an action such as removing 
access point information from WiFi scans and network traffic volume 
information in home networks, only allow to access ping and traceroute 
functionality.