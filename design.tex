\chapter{\sysname DESIGN}
\label{sec.design}
Our goal is to build a secure network testbed that can mitigate the problem 
of end user's security and privacy issues. 

The previous works introduced in \S{\ref{ssec.related_work}} describe features of current network testbeds. Our findings in \S{\ref{sec.relatedwork_motivation}} suggest that it is important to balance a trade-off between security and functionality.

In this section, we use this idea to guide our new design for building a secure network testbed. Our goal is to allow researchers to run experiments on home networks while reducing the risk of security and privacy problems.

\section{Overview}
Today's lack of knowledge about home networks hampers progress in a number of important research areas, from WiFi management strategies to broadband characterization. Previous works~\cite{sanchez2014measurement,dhawan2012fathom} usually provide a programmable interface for writing and launching measurements from the perspective of end systems. Our proposed approach in supporting a wide range of measurements uses a programmable interface as well. Unlike such prior network measurement efforts, our work embeds network measurement capabilities into a home access point. This approach has various unique advantages including: (i) our platform is able to observe all traffic to and from its clients; and, (ii) the home access point is a ``demarcation point'' between the home wireless network and the access link so that it is the best place to learn about wired and wireless network. Although BISmark~\cite{183951} has worked on deploying measurements through gateways in homes, it has several security issues that might cripple a user's Internet connection (e.g., malicious codes might run out of a router's resource). Instead, \sysname allows untrusted parties to run any experiment code on our platform via a programmable interface and resource protection framework. A user can let a third party have access to his network traffic data within a secure and performance isolated virtual machine. Users not only are able to control how much information they share with the rest of world, but also decide how many resources an application can use. For example, an experiment could be terminated if it consumes too many resources or accesses specific interfaces not allowed by a device's owner. 

\sysname incorporates three separate roles: a home wireless router, the \sysname provider and an experimenter. The router owner donates computational resources to our platform, while the \sysname provider enables experimenters to pool and share resources, The experimenter seeks to run experiments on a remote router.

Before any network measurement experiments can be conducted, a device owner first downloads an installer package from the \sysname provider. The latter is a testbed server that keeps track of routers and allows researchers to access them when available. To run experiment code on an owner's router, the experimenter needs to download an experiment manager to his own laptop. He or she uses the experiment manager to access the remote router directly, to upload experiment codes, and to start or stop the execution of the experiment. Figure~\ref{fig-arch} describes the different components and their interactions with \sysname.

\begin{figure}%[h]
\centering
\includegraphics[width=0.8\columnwidth]{figure/soar-arch.png}
\caption{\sysname architecture and components.}
\label{fig-arch}
\end{figure}

\section{Platform Architecture}
The components in \sysname have three possible roles: the home wireless router (where an experiment runs), the platform provider (that distributes an installer containing an experimenter's credentials) and the experiment manager (where experiments are initiated).

\subsection{Home Wireless Router}
Home wireless routers provide computing resources for experimenters to use in their network measurements. To isolate experiment code from the router, the code is executed in a \textbf{sandbox} environment. The default sandbox in \sysname is \sandboxname, a subset of the Linux Kernel API constructed with the Restricted Python sandbox. When experimenters want to deploy their code on a remote router, they need a way to upload code into a sandbox and start it. The \textbf{node manager} listens for remote commands and mediates access to the sandboxes to ensure that only authorized experimenters can execute code. Finally, the \textbf{software updater} automatically pushes any updates to the nodes instead of searching for new releases of the software manually.  

\subsection{Platform Provider}
The platform provider helps experimenters to pool and share platform resources. An experimenter can distribute a \sysname installer with his credentials inside and share it with router owners. To join \sysname, a router owner first acquires a \sysname installer. The sandbox, nodemanager, software updater and a set of public keys are packaged into this installer by a \textbf{package builder}. After an installation, the experimenter uses the public keys to register this router in a \textbf{lookup service} so that this router will be discovered by other experimenters.

\subsection{Experiment Manager}
Experimenters use their local machines to initiate and control experiments on a router running \sysname. The \textbf{experiment manager} allows experimenters to deploy code into the sandbox and debug the result on a router running \sysname. The experiment manager first uses the lookup service to find the set of routers running \sysname that are associated with an experimenter. It then communicates with these routers, executing commands on the behalf of the experimenter.  

\section{API Design}
\sysname is a network measurement platform designed to facilitate a broad range of experiments on a home gateway while controlling the impact on hosts' resources and network connectivity. A key challenge is selecting a programming interface that is both flexible (i.e., supports a rich set of network measurements) and safe (i.e., does not disrupt a normal user's Internet connectivity). Previous works on sandboxed programming measurement environments, including the Seattle Testbed, suggested Repy could serve as a useful environment for running network experiments. Considering our goal of supporting a wide range of network measurements on a home wireless router, we opt for extending a few capabilities of the Repy sandbox~\cite{cappos2010retaining}. The major goals of the sandbox are to provide security and performance isolation, and a portable programming interface across diverse device types. Currently, this programming interface provides functions for networking, file system access, threading, locking, logging, and so on. These cover a broad range of network and file I/O capabilities. Repy is highly portable, and has been used in many network measurement projects, including Content Distribution Network (CDN) Measurements~\cite{rafetseder2011exploring}, Video Streaming and Overlay Routing~\cite{eisl2011service}, Geographic Visualization of Mobile Network Data on open street maps~\cite{open3gmap}, and so on. Table~\ref{table:new_api} and \ref{table:Repy} provide a summary of our proposed API calls and the current set of measurement capabilities Repy supports. 

Network measurement platforms in today's home networks should support both active and passive measurement. Active measurements require sending packets to connected home devices. There are some Active measurement tools, such as Ping~\cite{ping} and Traceroute~\cite{traceroute} were used to learn about round-trip delays and network topologies. In comparison to active measurement, passive measurements do not create or modify any traffic on the network. They watch the traffic and their corresponding timestamps are transmitted by the platform. These tools can provide valuable information about network trouble-shooting and user behavior patterns. We now briefly describe these API calls corresponding to network measurements. Detailed information about supported network measurements will be described in \S{\ref{sec.network_measurement}}. \sysname achieves the implementation of these proposed API calls by adding the following functions to a sandbox (see \ref{sec.sandbox}). 

{\raggedright
\textbf{Basic Statistics.} \texttt{get\_network\_bytes}, \texttt{get\_network\_packets} and \texttt{get\_network\_interfaces} record aggregate network statistics for passively collected data. These methods read directly and parse data from \texttt{/proc/net/dev}. These are able to leverage the naturally-generated network traffic as passive measurements (particularly in the area of broadband characterization) by continuously monitoring the home gateway Internet connection.

\textbf{Wireless Information.} \texttt{scan}, \texttt{get\_station} and \texttt{wifi\_status} provide a Python interface to low-level system calls. The function sanitizes the call arguments, parsing complex output and returning results to the caller. These methods use the \texttt{iw} command-line tool to trigger a new WiFi scan and to retrieve its results. \texttt{iw} is a new nl80211 based CLI configuration utility for wireless devices. It supports all drivers that have been added to the kernel recently~\cite{iw}. They enable experimenters to study dense WiFi networks and home network characteristics.

\textbf{Measurement Tool.} \texttt{ping} and \texttt{traceroute} serve as the basis for active measurements. These tools can be combined to build a wide range of measurement experiments. They also are able to contribute to local network debugging. Because these are commonly used, they are better to support on cross-platform experiments. In order to let these measurement tools perform on diverse operating systems, we use a pure raw socket method for implementation. 


\begin{table*}
\scriptsize
\centering
\begin{tabular}{|p{.2\textwidth}| p{.6\textwidth}| m{1.5\textwidth}|}
\hline
\textbf{API}    &  \textbf{Description} \\
 \hline
 {\bf scan} & {\bf Collect the list of access points found with a WiFi scan. For each access point we collect BSSID, SSID, signal strength and channel number.} \\
\hline
 {\bf get\_station} & {\bf Record downlink statistics per associated client (e.g., Total packets sent, received, retried, client's signal strength in home wireless router).} \\
\hline
 {\bf wifi\_status} & {\bf Collect a list of WiFi channel information. (e.g., frequency, channel busy time, channel active time, channel receive time and channel transmit time).} \\
\hline
 {\bf get\_network\_interfaces} & {\bf Return a list of available network interfaces.} \\
\hline
 {\bf get\_network\_bytes} & {\bf Record information about the configured network interfaces. The statistics include metrics such total number of received or transmitted bytes, drops, errors.} \\
\hline
 {\bf get\_network\_packets} & {\bf Record information about the configured network interfaces. The statistics include metrics such total number of received or transmitted packets, drops, errors.} \\
\hline
 {\bf ping} & {\bf A pure python ping implementation using raw sockets.} \\
\hline
 {\bf traceroute} & {\bf Return the route packets take to network host. } \\
\hline
\end{tabular}
\caption {A summary of our proposed API calls.}
\label{table:new_api}
\end{table*}

\begin{table*}[]
\scriptsize
\centering
\begin{tabular}{|l|l|lll}
\cline{1-2}
Network                                                                                                                                                                                                                                                                                         & File system                                                                                                                                         &  &  &  \\ \cline{1-2}
\begin{tabular}[c]{@{}l@{}}gethostbyname\\ getmyip\\ sendmessage\\ openconnection\\ socket.close\\ socket.send\\ socket.recv\\ listenforconnection\\ tcpserversocket.getconnection\\ tcpserversocket.close\\ listenformessage\\ udpserversocket.getmessage\\ udpserversocket.close\end{tabular} & \begin{tabular}[c]{@{}l@{}}openfile\\ close\\ readat\\ writeat\\ listfiles\\ removefile\end{tabular}                                                &  &  &  \\ \cline{1-2}
Threading                                                                                                                                                                                                                                                                                       & Miscellaneous                                                                                                                                       &  &  &  \\ \cline{1-2}
\begin{tabular}[c]{@{}l@{}}createlock\\ lock.acquire\\ lock.release\\ createthread\\ sleep\\ getthreadname\end{tabular}                                                                                                                                                                         & \begin{tabular}[c]{@{}l@{}}log\\ getruntime\\ randombytes\\ exitall\\ createvirtualnamespace\\ getresources\\ virtualnamespace.evalute\end{tabular} &  &  &  \\ \cline{1-2}
\end{tabular}
\caption{Repy key API functions to support issuing and programming measurement experiments.}
\label{table:Repy}
\end{table*}

\begin{table*} 
\scriptsize
\centering
\begin{tabular}{|p{.1\textwidth}| p{.3\textwidth}| m{.3\textwidth}|}
\hline
\textbf{Type} & \textbf{Parameters} & \textbf{Descriptions} \\
 \hline
 {\bf Passive} & {\bf Aggregate traffic statistics per associated client (e.g., Total packets sent, received, retried,
client\'s signal strength at AP)\newline nearby APs information \newline Channel utilization} & {\bf Home network usage patterns\newline Wireless network performance} \\
\hline
 {\bf Active} & {\bf Throughput, Latency, Loss, Jitter \newline traceroute \newline DNS lookups} & {\bf ISP performance \newline Large-scale topology mapping} \\
\hline
\end{tabular}
\caption {Measurement types supported by \sysname.}
\label{table:experiment}
\end{table*}

\section{Supported Network Measurements.}
\label{sec.network_measurement}
We begin by presenting examples of measurement efforts that benefit from the platform location (at the hub of a home network) and supported capabilities. The list here is far from exhaustive; our design is general, and, hence, adaptable to a range of other experiments as well. Table~\ref{table:experiment} shows the type of network measurements obtained from \sysname.

\begin{itemize}
\item \textbf{Wireless network performance:} Unlike prior wireless measurement studies that have deployed passive monitors~\cite{mahajan2006analyzing,raghavendra2009wi,papagiannaki2006experimental}, in our work, we collect wireless performance metrics using home wireless routers as vantage points. Currently, \sysname supports gathering a variety of wireless network metrics, such as signal strength, tx/rx bitrate, SSID, BSSID, channel number, channel busy time. Unlike out-of-band wireless monitors, our wireless access points can exactly observe all traffic to and from its clients, correlate them with wireless performance, and conduct multiple types of measurements. In addition, a home router sits between the access link and wireless network so that it is better able to identify and isolate problems between these two locations.

\item \textbf{ISP performance:} The home wireless router is ideally suited for measuring access link characteristics without being affected by confounding factors from the rest of the home network. We support a wide range of network I/O primitives that can perform active measurement from home routers to a server, including speed tests, performance diagnostics and so on. For example, it can download small binary files through a TCP connection from a web server to a router to estimate the connection speed.

\item \textbf{Large-scale topology mapping:} Many prior works from the research community have tried to understand the Internet's topology and connectivity by conducting \texttt{traceroute-like} measurements from multiple vantage points~\cite{paxson1996end,chen2009sidewalk}. A router-based platform that supports \texttt{traceroute} functionality would help researchers to extend their study.

\item \textbf{Home network usage patterns:} Since our platform sits between the access link and the rest of the home network, and acts as a continuous monitoring software, we can study home network usage patterns, such as the amount of traffic and the number of devices on the network at any given time. For instance, \texttt{get\_station} is able to gather information about the number of connected devices and \texttt{get\_network\_bytes} has the ability to know the amount of network traffic.    
\end{itemize} 
\par}
\section{Security}
\label{sec.security}
With the rapid development of smart home technology, it becomes essential to focus on home network security and privacy. Since \sysname deploys on home wireless routers, any bugs can have a significant impact on a home network. For instance, attackers may obtain control of home networks through a malicious attack. 

To understand how to prevent attacks, we first considered what different kinds of attacks are possible. We observe that they fall into three types: (i) Denial of service attack (e.g., Packet flooding, SYN flooding~\cite{eddy2011syn}): overwhelm targets with a flood of network traffic to drain end host resources or control/drain resources on routers; (ii) Injection attack (e.g., Remote exploits~\cite{shellcode},  spoofing attack~\cite{bishop1996attack}): create a few ``magic'' packets to implement spoof attacks or exploit attacks; (iii) Insider attack (e.g., Network eavesdropping): malicious insiders intentionally eavesdrop, steal, or damage information to violate user's privacy.

To prevent these types of attacks, \sysname uses a sandboxed environment named Repy for security isolation and performance isolation.
 
\begin{itemize} 
\item \textbf{Security isolation:} The sandbox must be able to ensure the safe execution of experiment code, while presenting as little risk to the host as possible. The Repy sandbox in \sysname uses language-based isolation and is implemented in Python. In addition, Repy uses a security layer to push library functionality out of the sandbox kernel, thereby helping to mitigate the impact of bugs in libraries~\cite{cappos2010retaining}. As a result, an experimenter can enforce a policy upon a Repy sandbox without needing to change the TCB (trusted computing base).  
\item \textbf{Performance isolation:} Resource consumption is the biggest issue for a distributed system. In the home network, we need to share our platform with an unidentified home wireless router. If vulnerabilities come into effect\cite{joshi2013survey}, the rate of resource consumption may degrade the performance itself, and the performance of other applications in real networks~\cite{joshi2013survey}. Therefore, \sysname must control the load its experiments impose on the routers, as well as minimize the impact of real user's network connectivity. To achieve this, \sysname should limit consumption of host's resources, including the amount of CPU and memory used, network and disk I/O. To control CPU and memory utilization, \sysname monitors the amount of CPU and memory available to a experiment. To control network and disk I/O, \sysname checks calls that access these resources and prevents their execution if they exceed a predefined limit. 
\end{itemize}

Through performance isolation, the volume of an experiment's network traffic and the available resources for experiments is restricted based on the quota the router owner predefines. Therefore, denial of service attacks are impossible because our platform does not allow a sustained flood of traffic to launch to other hosts. In addition, since the sandbox controls CPU, memory, bandwidth, disk I/O and other resources, \sysname is able to control the load of its experiments imposed on a wireless router. Through security isolation, an experimenter has the ability to set up policies for libraries and returned data. For instance, if one wants to restrict access to a specified functionality's parameter or returned value, this can be added in a secure policy without changing Repy. To mitigate the second and third types of attacks, we could block a few functionalities that may pose risks to home networks, and anonymize sensitive information that might leverage users' privacy. Detailed information is discussed in \S{\ref{sec.privacy}}. 

\section{Privacy}
\label{sec.privacy}
Although a rich set of capabilities enhance the convenience of user interfaces and application usefulness, they also raise serious privacy concerns. For example, network traffic volume over an interface and the information from connected devices in home networks produce a rich history that can lead to behavior prediction, which can invade user privacy. 

In order not to allow measurements to exfiltrate sensitive information from a home gateway, \sysname provides a framework of reference monitors~\cite{ref} to enforce mandatory access control of sensitive data in real time. Figure~\ref{fig-reference} shows an overview of the reference monitor. Instead of modifying the sandbox, \sysname allows users to decide not only what information experimenters can capture on their routers, but also how much information they can share with others. \sysname allows router owners to add the privacy filters they wish to apply. These may range from sharing some home network device information with an experimenter by just blocking MAC addresses, to completely denying access to network traffic statistics. For a particular experiment, a privacy policy might be based on the type of network measurement, and the capabilities can be passed through, filtered, or dropped by the reference monitor. For an experimenter who wants to learn about Internet censorship, a filter might perform an action such as removing access point information from WiFi scans and network traffic volume information in home networks, only allowing access to ping and traceroute functionality. These privacy filters are easily built and disseminated to router owners. Therefore, \sysname is able to handle the second and third types of attacks in \S{\ref{sec.security}}. 

\begin{figure}%[h]
\centering
\includegraphics[width=0.8\columnwidth]{figure/referencemonitor.png}
\caption{Overview of reference monitor. The original Repy interface and the added API together provide the OS level sandbox kernel on a home wireless router. Different policies can be combined together as a set of filters to protect users' privacy before an experiment can access data.}
\label{fig-reference}
\end{figure}

