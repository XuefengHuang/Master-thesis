\section{Related Work}
\label{sec.related_work}
Our work shares goals with and builds upon ideas from several prior large-
scale platforms targeting home network research.In the following paragraphs, 
we briefly review related works on network testbed. 

The BISmark~\cite{183951} project aims at understanding the performance of 
wired access networks in homes through the long-term deployment of gateways. 
We use a similar approach but focus on building a programming framework to 
balance between flexibility (allowing researchers to do a wide set of 
experiments) and a constrained programming environment (limiting researchers 
from doing experiments that could interfere with home users). The RIPE Atlas~\cite{ripeatlas} is a global network of probes that measure Internet 
connectivity and reachability, providing an unprecedented understanding of 
the state of the Internet in real time. It has deployed thousands of probing 
devices worldwide, but their capabilities are limited to simple measurements 
(e.g., ping, traceroute). Our testbed provide a wide range of capabilities 
that support the implementation of as wide an array of network measurements 
as possible. SamKnows~\cite{samknows} has designed and developed its 
performance tests in house, adhering to IETF RFCs where appropriate. All 
measurements are written in C, for performance and portability across a 
range of hardware platforms. But it cannot adapt their measurements 
procedure and software in the course of a measurement study. Seattle instead 
allows for very general programmability and rapid code updates on wireless 
router. BeHop~\cite{yiakoumis2014behop} is a wireless testbed for dense WiFi 
networks often seen in residential and enterprise settings. Seattle not only 
evaluate WiFi management strategy, but also study different aspects of home 
networks, including access link performance, user behavior patterns and 
topology and connectivity characterization. WiSe~\cite{patro2013observing} 
is a measurement and management infrastructure that uses APs as vantage 
points to collect comprehensive information about network. However, WiSe 
focuses solely on the wireless network properties and not on the rest of the 
ISP path. Seattle instead focus on both characterization of the wired 
Internet path from the ISP’s network into the home and the wireless network 
properties.

Prior works such as~\cite{sanchez2014measurement},~\cite{dhawan2012fathom}
,~\cite{kreibich2010netalyzr} has focused on end-host techniques for 
measuring end to end performance in home networks. Dasu~\cite{
sanchez2014measurement} is a host-based software that analyzes BitTorrent 
traffic to characterize ISP performance. Because Dasu piggybacks on 
BitTorrent, it has a large user base.Fathom~\cite{dhawan2012fathom} is a 
Firefox extension that provides a programmable interface for writing and 
launching measurements from the convenience of the web browser. Netalzyr~\cite{kreibich2010netalyzr} also lets users conduct a series of tests from a 
browser. However, these testbeds cannot support for continuous measurements (
i.e., since hosts can be turned off, moved, etc.). In additional, they do 
not reflect the performance of a fixed network vantage point due to the 
limitation of host. 
