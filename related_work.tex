\section{Related Work}
\label{sec.related_work}
Our work shares goals with and builds upon ideas from several prior large-
scale platforms targeting home network research.In the following paragraphs, 
we briefly review related works on network testbed. 
\begin{itemize}
\item \textbf{BISmark}~\cite{183951} is similar to Seattle in that it is 
deployed in home networks on resource-constrained devices. And currently it 
is used and shared by researchers at nine institutions, including commercial 
Internet service providers, and has enabled studies of access link 
performance, network connectivity, Web page load times, and user behavior 
and activity. However, vetting experiments is challenging. A poorly designed 
(or controlled) experiment can cripple a user's Internet connection. Seattle 
is supported to run arbitrary codes on remote devices without compromising 
the security and privacy of device owners.
\item \textbf{Samknows} has designed and developed its performance tests in 
house, adhering to IETF RFCs where appropriate. All measurements are written 
in C, for performance and portability across a range of hardware platforms. 
But it only supports limited performance measurements. Seattle supports a 
very flexible language for experiment specification based on a restricted 
subset of Python.
\item \textbf{The RIPE Atlas project}~\cite{bajpai2014lessons} is a global 
network of probes that measure Internet connectivity and reachability, 
providing an unprecedented understanding of the state of the Internet in 
real time. It has deployed thousands of probing devices worldwide, but their 
capabilities are limited to simple measurements (e.g., ping, traceroute). 
Seattle instead provides a wide range of capabilities to support complex 
measurements. 
\item \textbf{Dasu}~\cite{sanchez2014measurement} is a measurement 
experimentation platform for the Internet edge. It supports both controlled 
network experimentation and broadband characterization. But it is not able 
to run a range of measurements due to its capabilities are limited and 
cannot run continuous measurements (i.e. , since hosts can be turned off, 
moved, etc.).
\item \textbf{Scriptroute}~\cite{spring2003scriptroute} provides a safe, 
publicly-available network probe execution environment, but ScriptRoute 
hasn't been updated in a few years. Furthermore, it doesn't support IPv6 and 
TCP options. Currently, Seattle not only engage in research projects, but 
also  allow students to get practical experience with real-world end user 
networks.
\item \textbf{Fathom}~\cite{dhawan2012fathom} project explores the browser 
as a platform for network measurement and troubleshooting. It provides a 
wide range of networking primitives directly to in-page javaScript. You get 
direct TCP/UDP socket access, higher-level protocol APIs such as DNS, HTTP, 
and UPnP, and ready-made functionality such as pings and traceroute. 
However, the measurements are not continuous(i.e. , since browser can be 
turned off, etc.). Seattle instead is deployed on home routers so that 
measurements are continuous, direct and comprehensive.
\item \textbf{HomeNet Profiler}~\cite{dicioccio2013measuring} is a software 
that runs on any computer connected inside a home network, to collect a wide 
range of measurements about networks including the set of devices, the set 
of services and the characteristics of the WiFi environment. But due to it 
lacks programming interface so HomeNet Profiler is not flexible. Its 
measurements are also not continuous. Seattle not only support these 
capabilities HomeNet Profiler has, but also allow more measurement primitives
(traceroute, ping, etc...) while preserving security and privacy.
\item \textbf{Google Analytics} is an enterprise-class based web analytics 
tool which provides a transparent view of website traffic and marketing 
effectiveness. Google Analytics has powerful and advance features that give 
rich insight into the websites and improve website ROI (Return on Investment)
. But it only supports limited website performance measurements.
\item \textbf{IETF's LMAP} is a desire to be able to coordinate the 
execution of broadband measurements and the collection of measurement 
results across a large scale set of Measurement Agents (MAs). These MAs 
could be in every home gateway and edge device such as set-top boxes and 
tablet computers, and located throughout the Internet as well. The 
measurement system is under the direction of a single organisation and each 
MA may only have a single Controller at any point in time.
\end{itemize}
