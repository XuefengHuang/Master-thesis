\chapter{RELATED WORK AND MOTIVATION}
\label{sec.relatedwork_motivation}
In this section, we present some background information on factors that motivated the design and deployment of \sysname. First, we look at related works behind the design. Then we talk about the motivating factors that shaped our choices.
\section{Related Work}
\label{ssec.related_work}
Our work shares goals with, and builds upon, ideas from several prior large scale platforms targeting home network research. In the following paragraphs, we briefly review related works on existing research platforms and home network experiments.  

\textbf{Existing research platforms. }The BISmark~\cite{183951} project aims at better understanding the characteristics of wired access networks in homes through continuous measurements from long-running home gateways. We use a similar approach but focus on building a programming framework to balance between flexibility (allowing researchers to do a wide set of experiments) and a constrained programming environment (limiting researchers from doing experiments that could interfere with home users). The RIPE Atlas~\cite{ripeatlas} is a platform that measures Internet connectivity and reachability. It has deployed thousands of probing devices worldwide and provides a better understanding of the Internet in real time. However their capabilities are limited to simple measurements (e.g., ping, traceroute). Our platform provides a wide range of capabilities that support the implementation of as wide an array of network measurements as possible. SamKnows~\cite{samknows} helps researchers measure broadband performance by installing hardware in their homes. To provide better performance and portability across a range of hardware platforms, it chooses the C programming language to write all measurements. But it cannot adapt their measurements procedure and software over the course of a study. \sysname instead allows for very general programmability and rapid code updates on a wireless router. 

BeHop~\cite{yiakoumis2014behop} is a wireless testbed that aims to provide insights on dense WiFi networks in residential and enterprise settings. \sysname not only evaluates WiFi management strategy, but also studies different aspects of home networks, including access link performance, user behavior patterns and topology, and connectivity characterization. WiSe~\cite{patro2013observing} is a residential deployment that collects information about networks using access points as vantage points. However, WiSe focuses solely on the wireless network contents and not on the rest of the ISP path. \sysname instead focuses on both characterization of the wired Internet path from the ISP\'s network into the home and on wireless network properties.

Prior works such as~\cite{sanchez2014measurement,dhawan2012fathom,kreibich2010netalyzr} focused on end-host techniques for measuring end to end performance in home networks. Dasu~\cite{sanchez2014measurement} is a host-based software that analyzes BitTorrent traffic to measure ISP performance. It has many users that benefits from deployment on BitTorrent. Fathom~\cite{dhawan2012fathom} is a Firefox extension platform and it provides a programmable interface for writing and collecting measurements from the web browser. Netalzyr~\cite{kreibich2010netalyzr} also lets users conduct many network measurements from a browser. However, these testbeds cannot support continuous measurements (i.e., since the host can be turned off, moved, etc.). In addition, they do not reflect the performance of a fixed network vantage point, due to the limitation of the host. Our work extends Seattle Testbed~\cite{cappos2009seattle}, which runs on laptops, tablets and smartphones. In contrast to Seattle Testbed, our platform is deployed in home networks on resource-constrained devices and extends eight capabilities based on the Seattle sandbox.

Most active measurement and experimental research relies on dedicated infrastructures (PlanetLab~\cite{chun2003planetlab}, MLab~\cite{mlab}). Although these infrastructures support long-running measurements, these nodes are not representative of the real users' Internet experience because their deployments are in managed settings like academic or research networks. In \sysname, experiments can observe all traffic passing through a home wireless router. Scriptroute~\cite{spring2003scriptroute} provides a sandboxed environment for any researchers to run experiments safely in PlanetLab. We share Scriptroute's goal of building a community platform for network measurements. However, \sysname is intended for large scale deployment on home wireless routers, which brings new challenges, particularly resource limitations and home network security.

\textbf{Existing home network experiments.} There has been significant interest in measuring home networks over the past ten years. Previous studies~\cite{aguayo2004link,kotz2005analysis,raman2009feasibility} used passive measurements to evaluate the performance of 802.11 wireless networks in campus, enterprise, urban and rural environments. Raman et al.~\cite{raman2009feasibility} undertook a detailed measurement study for mesh networks in rural locations. It provided evidence to support a conclusion that external interference is the main cause of unpredictable link behavior. Kotz et al.~\cite{kotz2005analysis} conducted a trace-based study of WLAN users on the Dartmouth College campus, in an effort to understand patterns of activity. However, this approach does not allow researchers to understand performance experienced in all environments. For example, when network throughput changes suddenly, it is difficult to infer whether it is due to signal strength or only from wireless congestion from passive measurements.

Recent studies have tried to measure home networks in various ways, from the end-host \cite{chetty2011my,dicioccio2012probe,sanchez2013trying}, or from gateway routers \cite{grover2013peeking,sundaresan2015measuring,patro2013observing,sundaresan2013measuring,pefkianakis2015characterizing}. For example in \cite{sundaresan2015measuring}, authors deploy BISmark in many different homes to study the relationship between wireless metrics and TCP performance of user traffic. Patro et al.~\cite{patro2013observing} deploy WiSe to perform wireless network measurements in different homes and propose a simple metric to estimate TCP throughput. S{\'a}nchez et al.~\cite{sanchez2013trying} explore the environment of home networks around the world through UPnP (Universal Plug and Play) measurements collected from Dasu. Sundaresan et al.~\cite{sundaresan2013measuring} identify potential factors that affect Web performance bottlenecks through the deployments of FCC/SamKnows and BISmark. We share the idea of Srikanth et al.~\cite{sundaresan2015measuring} to study how home wireless performance characteristics affect network performance on user traffic in real homes. However, we also study wireless performance in concert with channel utilization.

\section{Motivation}
\label{sec.motivation}
A primary motivation for this study was to be able to run network measurements while ensuring device security and user privacy. The list here is a unique set of challenges when deploying a testbed in home networks. Our design and deployment is in home networks, and hence addressing these issues as well.

\begin{itemize}
\item \textbf{Protecting User Privacy:} \sysname runs on personal wireless routers, which have the potential for violations of privacy if a device owner's network activity and personally identifiable information were to become public. For instance, data acquired through programming interfaces can be highly private, exposing sensitive information, such as when a user is at home and using the network. We find that it was hard to strike a balance between the need for open data and user privacy during BISmark's development. Therefore, it is necessary to provide a dynamic and flexible access control mechanism to offer effective privacy protection. In this work, we introduce a security layer (see \S{\ref{sec.privacy}}) as one privacy protection mechanism. It is flexible to decide what information is captured on devices and how much information they could share with the public. 

\item \textbf{Protecting User Devices:} For a distributed system, it is necessary to control resource consumption on the local host and minimize the impact of Internet connectivity. As such, if an attacker consumes a large amount of resources, it can lead to poor performance. Therefore, managing resources on a large scale while providing performance isolation is a key challenge for any distributed system software. Many network testbeds consider this issue when designing their platform. For instance, all experiments on Scriptroute~\cite{spring2003scriptroute} are executed in a resource-limited sandbox to protect platform from resource abuse and launching denial-of-service attacks. Dasu also uses a sandbox environment for defining and deploying experiments, and controlling their malicious impact. However, they do not have practical experience on home wireless router. In our work, we propose a secure sandbox running on a home wireless router to provide security isolation and performance isolation. From our practical deployment, we find this is an useful environment for running a wide range of experiments while protecting devices.

\item \textbf{Protecting Home Networks:} There is an important difference between \sysname and other testbeds, such as PlanetLab, MLab. Our experiments involve home networks. Attackers can get the control rights of a home network through our platform and send malicious control commands to different kinds of WiFi enabled devices such as laptops, printers, wireless based security cameras, entertainment devices and other security systems, which may lead to undesired consequences. By exploiting vulnerabilities in home networks, attackers also can gather information on targets, threatening home network's privacy and safety, understanding behavior and patterns. For instance, some devices can be shut down unusually or the information of some devices can be monitored. To address these issues, we summarize potential network attacks and the solutions to these problems (see \S{\ref{sec.security}})
\end{itemize}

