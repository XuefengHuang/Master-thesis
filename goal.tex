\chapter{CHALLENGES AND GOALS}
\label{sec.goals_challenges}
\section{Challenges}
\label{ssec.challenges}
A testbed that allows researchers to execute code across donated embedded devices faces four major challenges: 
\begin{itemize}
\item First, the embedded devices pose a limited resources challenge. It is hard to run heavy scripting languages, such as Python or Ruby. Unlike host-based or fixed server network testbeds, router-based network testbeds deploy on resource-constrained devices that have too little RAM and flash storage. Therefore, our platform must be small and easy-to-deploy. To overcome this challenge, BISmark uses standard UNIX utilities and small C programs with shell and Lua scripts.  
\item The second challenge is preserving a user's network experience. \sysname is on the direct path of real Internet users. A malicious network measurement experiment will noticeably affect a normal user's network experience by consuming too many network resources. Therefore, our platform must ensure that a poorly designed experiment cannot disrupt a user's Internet connectivity. This presents a challenge for systems like BISmark. To fix this issue, it has to manually review each experiment's codes. Dasu uses a sandboxed environment for safe execution of experiment and limits resource consumption. However, it uses Drools (Dasu's when-then language) so we need to port from Java.
\item The third challenge is to reconcile the need for open data with user privacy. Since a home router connects to many devices in a home network, they can generate valuable data for a home network research study. However, this data can also threaten user privacy. For example, network traffic statistics can yield insight into user behavior and indicate when users are home. Due to these concerns, network testbeds either do not collect data from real Internet users or limit data collection to a controlled group of participants. Home users want information security and researchers want to gather data and information. Therefore, it is difficult to balance data collection with privacy protection.
\item Further challenges remain regarding software upgrades. Unlike PlanetLab nodes deployed in universities, network testbeds in home networks cannot get technical support. In addition, OpenWrt's built-in \texttt{opkg} lacks features for managing software in the homes of non-technical users. Therefore, software management is a big challenge for \sysname.  
\end{itemize}
In the past, Python was relatively difficult to be installed on home routers because of limited free space on flash storage. With the further development of hardware technology, and what was learned from our previous practical experience, we know running Python code on embedded devices meet the first challenge. We ported \sysname onto a home wireless router, as described in \S{\ref{ssec.deployment}} and ran many network measurements (including multithreading, network programming, file handling, etc) as documented in \S{\ref{sec.evaluation}}. The second and third challenges can be solved by using a secure and performance-isolated sandbox (see \S{\ref{sec.sandbox}}) and a reference monitor framework (see \S{\ref{sec.policy}}). Finally, \sysname includes a software updater (see \S{\ref{sec.softwareupdater}}) to ensure system robustness.

\section{Goals}
\label{ssec.goals}
The goal of \sysname is to provide a scalable platform for running meaningful experiments on home networks. It is designed to achieve:
\begin{itemize}
\item Safety: Internet users who participate in the testbed should not face significant risk of damage to their devices or internet service. Code should be strictly sandboxed. Code executed on the testbed must not interfere with the performance of the user's Internet connection. As mentioned in \S{\ref{sec.motivation}}, there are a few potential risks in home networks, therefore we design a network measurement platform with user security and privacy as first-order concerns.

\item Privacy: Our goal is striving to maximize utility to experimenters while not compromising the privacy of the user (e.g., expose home network usage behavior).

\item Flexibility: The related works in \S{\ref{ssec.related_work}} show that an effective network measurement platform should support both passive and active measurements. Passive measurements provide precise and continuous understanding of the network, while active measurements provide some insight into the way real traffic is treated within it. Therefore, we require a wide range of APIs that support the broadest spectrum of network measurements possible.

\item Lightweight: We desire the platform's weight to be as light as possible for two reasons. First, because \sysname is deployed in home networks on resource-constrained devices, it is impossible to run complex software. And, second,  \sysname is on the direct path of real Internet users. Hence, we do not want to impact users' Internet connectivity.

\item Accuracy: We desire the measurement platform to accurately track 
general network information. Precise experiment data is able to help researchers better understand the network.

\item Ease of installation and use: Unlike computer or mobile environments, many people are not familiar with home wireless routers. Therefore, installation is a big challenge. We require that the router be easy to install, un-install, and stop.
\end{itemize}

\sysname's design (\S\ref{sec.design}) addresses the first three goals, and its deployment experience and evaluation(\S\ref{ssec.deployment} and \S\ref{sec.evaluation}) the remaining goals.