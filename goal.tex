\section{Challenge and Goal}
\label{sec.goal}
A testbed that allows researchers to execute code across donated embedded 
devices faces three major challenges: First, the embedded devices poses a 
limited resources challenge - it is hard to run heavy scripting languages 
like Python or Ruby. The second challenge is user's networking experience. 
Seattle nodes are on the direct path of real Internet users. A malicious 
network measurement experiment will noticeably affect a normal user's 
network experience. The third challenge is secure data access - network 
traffic data pose a risk to device donors whose devices are exposed to 
malicious code written by users. The fourth challenge is data privacy - 
users should not gain access to information that donors don?t want to share. 
As the development of hardware technology and based on our practical 
experience, running Python codes are going well on embedded devices. And the 
last three challenges can be solved via using a secure and performance-
isolated sandbox and a reference monitor framework we proposed.

We identify the design goals for network measurement platform in home 
networks:
\begin{itemize}
\item Safety: Internet users that participate in the testbed should not face 
significant risk. Code should be strictly sandboxed. Code executed on the 
testbed must not interfere with the performance or correctness of user's 
Internet connection.
\item Extensibility: The platform should provide a wide range of APIs that 
support the implementation of as wide an array of network measurements as 
possible.
\item Lightweight: Due to Seattle is deployed in home networks on resource-
constrained devices, hence we don't want to impact users' Internet 
connectivity.
\item Accuracy: We desire the measurement platform to accurately track 
general network information. 
\item Easy to install and use: Should be easy to install, un-install, and 
stop. 
\item Portability: Measurement code should work portably on any 
implementation of the platform.
\end{itemize}

